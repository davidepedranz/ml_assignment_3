\section{Conclusion}
\label{sec:conclusion}

None of the proposed modification achieved to improve the accuracy of the original network.
The results are summarized in table \cref{tab:summary}.
The two layers of convolution were able to extract better features then the single layers of models 1 and 2.
Both the final performances and their evolution over the training epochs (\cref{fig:performances}) of all modified models are very closed to each other.
Model 1 seems to perform slightly better then the other two, even though some more experiments are required in order to get a confident result.
The worse performances of model 2 and 3 suggest the need of to compute very complex features in order to achieve a very high score on the image classification.


\begin{table}
	\centering
	\caption{Summary of the accuracies on the test set of the different networks at different epochs}
	\label{tab:summary}
	\begin{tabular}{lccc}
		\toprule
			\multicolumn{1}{l}{Network} &
			\multicolumn{1}{c}{Epoch 10000} &
			\multicolumn{1}{c}{Epoch 20000} &
			\multicolumn{1}{c}{Epoch 30000} \\
		\midrule
			Original & 99.11 \% & 99.21 \% & 99.22 \% \\
			Model 1  & 98.68 \% & 98.94 \% & 99.07 \% \\
			Model 2  & 98.75 \% & 98.96 \% & 99.02 \% \\
			Model 3  & 98.51 \% & 98.86 \% & 99.03 \% \\
		\bottomrule
	\end{tabular}
\end{table}
