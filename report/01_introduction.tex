\section{Introduction}
\label{sec:introduction}

Neural Network is a powerful learning algorithm widely used in Machine Learning to solve complex tasks like handwritten text recognition, computer vision, speech recognition and machine translation.
Indeed, Neural Networks are able to represent very complex functions, which make them very adaptable to many different problem.

Depending on the specific task to solve, Neural Networks with different structures can be built and trained.
In this assignment, we will start from the architecture suggested in the TensorFlow\textsuperscript{TM} tutorial \textit{Deep MNIST for Experts} \footnote{\url{https://www.tensorflow.org/versions/master/tutorials/mnist/pros/}} to recognize the handwritten digits of the MNIST \footnote{The MNIST dataset is available at: \url{http://yann.lecun.com/exdb/mnist/}} dataset, then remove one layer at a time and study the behaviour of the new network.
We will explain the structure of each network and the function of each layer.
Finally, we will compare the performances of the different networks.

There are many different libraries available to construct and train Neural Networks.
In this assignment, we will use TensorFlow\textsuperscript{TM}, an open source library for numerical computation using data flow graphs.
It allows to easily define the structure of a Neural Network and efficiently train it.
It was originally developed by the Google Brain Team and is now widely used by many students, researches and companies across the world.
